% This is samplepaper.tex, a sample chapter demonstrating the
% LLNCS macro package for Springer Computer Science proceedings;
% Version 2.21 of 2022/01/12
%
\documentclass[runningheads]{llncs}
%
\usepackage[T1]{fontenc}
% T1 fonts will be used to generate the final print and online PDFs,
% so please use T1 fonts in your manuscript whenever possible.
% Other font encondings may result in incorrect characters.
%
\usepackage{graphicx}
% Used for displaying a sample figure. If possible, figure files should
% be included in EPS format.
%
% If you use the hyperref package, please uncomment the following two lines
% to display URLs in blue roman font according to Springer's eBook style:
%\usepackage{color}
%\renewcommand\UrlFont{\color{blue}\rmfamily}
%\urlstyle{rm}
%
\begin{document}
%
\title{Challenges and Solutions of Developing and
Implementing a Revolutionary Novel Desktop-as-a-Service}
%
%\titlerunning{Abbreviated paper title}
% If the paper title is too long for the running head, you can set
% an abbreviated paper title here
%
\author{Christian Baun\orcidID{0009-0004-9955-3752} \and
Johannes Bouché\orcidID{TODO}}
%
\authorrunning{C. Baun, J. Bouché}
% First names are abbreviated in the running head.
% If there are more than two authors, 'et al.' is used.
%
\institute{Faculty of Computer Science and Engineering, Frankfurt University of Applied Sciences,\\
Nibelungenplatz 1, 60318 Frankfurt am Main, Germany
\email{[christianbaun|johannes.bouche]@fb2.fra-uas.de}}
%
\maketitle              % typeset the header of the contribution
%
\begin{abstract}
The paper will cover very novel and non-published knowledge we gained
from developing and implementing a DaaS that is funded by the German
Federal Ministry for Economic Affairs and Climate Action. The topic
fits very well into the topics of the conference. TODO: Christian

\keywords{DaaS \and Compatibility \and Performance \and Stability \and Usability}
\end{abstract}
%
%
%
\section{Introduction}

Inkl. Motivation

TODO: Christian (1)
TODO: Johannes (2)

\section{Related Work}

TODO: Christian (1)
TODO: Johannes (2)

\section{Architecture}

TODO: Christian (1)
TODO: Johannes (2)

\section{Challenges and Solutions}

TODO: Christian (1)
TODO: Johannes (2)

\subsection{Compatibility}

TODO: Christian (1)
TODO: Johannes (2)

Anwendungen (Windows / Linux / MacOS)

Sound (via sound-deamon oder via Guacamole)  

Drucker (via cups oder via Guacamole) 

Externe Geräte (USB-Stick)


\subsection{Performance}

TODO: Johannes (1)
TODO: Christian (2)
TODO Christian: Latenz bei Cloud-Gaming Erfahrungen

Welche Performance-Paramter / Kriterien sind hier relevant für die Anwendung / Nutzung
Fokus: Bottlenecks, Möglichkeiten der Performance-Steigerung, Grenzen/Limits, die man akzeptieren muss

Latenz hängt auch immer von Entfernung und Übertragungsmedium ab. 

Skalierbarkeit: Stand und Möglichkeiten

Ceph-Dateisystem 

Datenbank SQlite aktuell. Perspektivisch MariaDB

Guacamole läuft 1x direkt auf dem Host aktuell. Läuft theoretisch auf jedem Host.



\subsection{Stability}

TODO: Johannes (1)
TODO: Christian (2)

Was ist hier relevant?

Verfügbarkeit des Service und der Benutzerdaten

Redundante Komponenten (Hardware und Software)

Welche Komponenten kann man sinnvoll redundant vorhalten?

Was sind die Vor- und Nachteile?

Auch Kosten?

Wo sehen wir aktuell in unserer Architektur Risiken für Dienstausfall und Datenverlust?

Features von Proxmox

Ceph-Dateisystem

Datenbank SQlite aktuell. Perspektivisch MariaDB

Guacamole läuft 1x direkt auf dem Host aktuell. Läuft theoretisch auf jedem Host. Proxmox kann sich darum kümmern?!

Docker haben/nicht haben / Vor- und Nachteile

Hier auch auf die Skalierbarkeit beziehen

TODO Johannes: Beschreibung zur Position der Komponenten, Möglichkeiten der Parallelisierung und Einschränkungen 

\subsection{Usability}

TODO: Christian (1)
TODO: Johannes (2)

Ideen und Konzepte zur Trennung von User und Admin-Ansicht

Reduktion der Komplexität der Bedienung durch Fokus auf notwendigste Interaktionsschritte

Desktop vs. einzelne Anwendung im Browser-Tab

Anpassung des Browser-Inhalts auf die Größe der Anwendung

Herausforderungen, Entwicklung, Implementierung, Lessons-learned

TODO Johannes: Erlebnisse der letzten 3-4 Wochen aufschreiben



\section{Conclusion}

TODO: Zusammen am Schluss

\section{Outlook}

TODO: Zusammen am Schluss

\section*{Acknowledgements}

This work was funded by the Federal Ministry for Economic Affairs and Climate Action
('\textsl{Bundesministerium f\"ur Wirtschaft und Klimaschutz}')
in the framework of the central innovation programme
for small and medium-sized enterprises
('\textsl{Zentrales Innovationsprogramm Mittelstand}').

We thank our project partners from Nuromedia GmbH for their support.
We especially thank Björn Goetschke, Mike Ludemann, Dario Savella, Holger Sprengel, and Rahul Tomar.


\section*{References}

% ---- Bibliography ----
%
% BibTeX users should specify bibliography style 'splncs04'.
% References will then be sorted and formatted in the correct style.
%
% \bibliographystyle{splncs04}
% \bibliography{mybibliography}
%

% \bibliography{\jobname_no_dashes_at_repeated_names,references}{}


\end{document}
